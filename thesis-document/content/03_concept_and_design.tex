\chapter{Concept and Design}
\label{cha:conceptanddesign}

This chapter outlines the key components of the implementation of environments in the MS.
The core components are users, environments, roles, and assets.
In essence the concept can be summarized as follows: users can be part of to multiple environments.
Environments hold Assets.
Users that are part of an environment can work on the assets that are stored in it.
What a user can do with an asset is determined by the roles that the user has in the
environment that the asset is stored in.
All other components that will be introduced will help to manage and enforce these relationships.

\section{Users}
\label{cha:conceptanddesign:users}
%NOTE: maybe change the sctructure of this paragraph a bit

Users represent individual people utilizing PROCEED.
A single person can have one or more users, but each user is intended for individual use.
To facilitate the exploration of the MS, without creating an account, users
can use the MS as guests.
% NOTE: maybe shift this down
Thus, we differentiate between authenticated users and guest users.
Guest users have the ability to transition to authenticated users whilst retaining their assets.
% Creating a Guest user doesn't require any data from the person creating it.

All users have a personal environment \ref{cha:conceptanddesign:environments:personal}
in which they can create and manage assets freely.
Authenticated users can also be part of organization environments \ref{cha:conceptanddesign:environments:organization}
where they can collaborate with other users.

%TODO: say what guests can do?

% NOTE: I think this goes in implementation
% \subsection{Accounts}
%
% Accounts represent a use's sign in method. 

\section{Folders}

Folders are nodes in a rooted tree structure with a name and a description.
Folders can contain other folders and assets.
At the moment writing, folders only support processes, but they could be extended to
support other types of assets like the ones described in \ref{cha:relatedwork:proceed-assets}.

Each environment will have a folder structure to store its processes.
The root folder is created when the environment is created.
Each root folder and all of its children are contained by one and only one environment \ref{cha:conceptanddesign:environments}.

Folders are intended to allow users to mirror the hierarchical structure of their
organization and of its projects.

% Since the root folder belongs to only one environment, and all folders are stored below
% one root folder, all folders belong to only one environment.
% Processes will then store a reference to the folder they're stored in.

% TODO: ask kai if I should explain rooted trees - would be nice to cookup a formal def


\section{Assets}

All assets within the MS \ref{cha:relatedwork:proceed-assets}
will be modified, so that each asset instance establishes a clear association with a single environment.
Processes will be contained within folders, explicitly indicating to what environment they
belong.
Other asset types will store a direct reference to their environment, without being
contained in a folder structure, this can be seen as a flat folder structure.

% The key principle is that every asset belongs to one and only one environment, and this
% association can be easily identified.

% All of the MS's assets will only change in one way, they will somehow linked to an
% environment and only one.
% For processes, they're contained in a folder, such that they clearly belong to one
% environment. for other assets, they will store a reference to the environment they belong
% to.
% The important part is that each asset belongs to exactly one environment, and this
% environment can be identified.

\section{Environments}
\label{cha:conceptanddesign:environments}

Conceptually environments are the data structure in which everything, other than users, is
stored.
Users aren't stored in environments as they can be a part of multiple environments, 
so they can't be contained in only one environment.
Instead, the MS stores memberships that specify that a user is part of an environment.
Everything else, assets and roles, belong to exactly one environment.

We distinguish between two types of environments: personal environments and organization
environments.

\subsection{Personal Environments}
\label{cha:conceptanddesign:environments:personal}

Personal environments are assigned to each user once they sign in. 
%NOTE: remove the part with owner
The user for which the environment is created, is the only member of this
environment, and is therefore called the owner.
No other users be a part of this environment.
%TODO: forward ref to folders
Personal environment only allow users to create and manage processes and folders,
other Features that the MS offers are disabled for personal environments and can only be
used in organization environments \ref{cha:conceptanddesign:environments:organization}.

\subsection{Organization Environments}
\label{cha:conceptanddesign:environments:organization}

Organization environments are intended to be used by organizations, thus they can have a
name, description and a logo.
Organization environments extend the feature set of personal environments, the enabled
resources \ref{cha:relatedwork:proceedroles:ms-resources-actions} for organization
environments can be determined when deploying a MS instance.

Organization environments can also have multiple Users that are part of it, these are
called members.


\section{Roles}

Before the implementation of this thesis, roles in the MS
\ref{cha:relatedwork:proceedroles} were global,
meaning that their permissions applied to all assets in the MS.
With the introduction of the folder structure in organization environments, this no longer makes sense.
Folders allow organizations to mirror their hierarchical structure, but this wouldn't be
entirely useful if roles were still global.
For this reason, roles can now be associated to a folder.
Roles that are associated with a folder cascade down the
folder structure, i.e. a role associated to a folder will also apply to all of its children.
Roles that aren't associated to a folder will continue to apply to all assets in the environment.
As roles are meant to mirror a users position in an organization, they're only available
for organization environments.

% TODOFIG: show how roles cascade

If a user's role allows him to view assets and is associated to a folder, then the user
also has the permission to view all parent folders of the folder the role is associated to.
But this is only restricted to the parent folders, not the contents of the parent folders.
This allows users to navigate the folder structure until they reach the assets they're
allowed to view and manage.

% TODOFIG: show how user can view path to his folder

\subsection{Default roles}

For each organization environment two roles will be created, which cannot be deleted and
cannot be associated to a folder:

\begin{itemize}
  \item \lstinline{@admin}: This role has all permissions for all assets in the
    organization environment and it is first assigned to the user that creates the organization environment.
    Only users with the \lstinline{@admin} role can add new users to this role.
  \item \lstinline{@everyone}: The permissions in this role apply to for all the users
    that are part of the organization environment. The permissions in this role start out
    empty, but can be modified.
\end{itemize}

