\chapter{Conclusion}
\label{cha:conclusion}

Building an application structure that supports multi-tenancy presents many challenges,

This thesis addressed the challenge of implementing multi-tenancy in the PROCEED
Management System (MS).

by introducing the concept of environments. These environments
offer isolated workspaces for users, supporting both personal and organizational use cases
while maintaining the flexibility and security required for cloud-based applications.


The shift to multi-tenancy required rethinking how assets and users are stored. 
We introduced personal environments, where a single user can manage assets privately, and
organization environments, where multiple users can collaborate on shared folders and
processes. 
Along with the environment concept, we adapted the existing role system so that
permissions now respect both the folder hierarchy and environment boundaries, ensuring
that users only interact with assets they are supposed to.

A major focus was to keep the underlying code changes minimal and 
to make the new abstraction straightforward to use from a developer perspective.
Enforcing access rules in the backend through CASL and providing helper functions for the
frontend were crucial steps to maintain security and consistency across the MS.
We also paid careful attention to the handling of guest users, giving them the option to merge
their data with fully authenticated accounts without losing the assets they created.

Overall, the work presented here lays the foundation for a robust multi-tenant experience in the PROCEED MS. It strikes a balance between introducing an entirely new concept—environments—and preserving the existing structure of the system so that both end users and developers can comfortably adapt to the new features.







\section{Outlook}
\label{cha:outlook}

Nowadays software is so advanced in multi tenant platforms, companies have come to expect
a lot of features from such software.
Other than the role system this thesis implements the bare minimum to provide such an
experience.
There are still many features that could be implemented to improve the expierience.

- sharing documents to members that arent part of the organization
- Support for user directives -> everyone doesn't have to create an account manually
  - here it could be also possible that an organization is created and all accounts from
    the user directive are automatically added 
-
