\chapter{Conclusion}
\label{cha:conclusion}

% Building an application structure that supports multi-tenancy presents many challenges,
%
% This thesis addressed the challenge of implementing multi-tenancy in the PROCEED
% Management System (MS).
%
% by introducing the concept of environments. These environments
% offer isolated workspaces for users, supporting both personal and organizational use cases
% while maintaining the flexibility and security required for cloud-based applications.
%
%
% The shift to multi-tenancy required rethinking how assets and users are stored. 
% We introduced personal environments, where a single user can manage assets privately, and
% organization environments, where multiple users can collaborate on shared folders and
% processes. 
% Along with the environment concept, we adapted the existing role system so that
% permissions now respect both the folder hierarchy and environment boundaries, ensuring
% that users only interact with assets they are supposed to.
%
% A major focus was to keep the underlying code changes minimal and 
% to make the new abstraction straightforward to use from a developer perspective.
% Enforcing access rules in the backend through CASL and providing helper functions for the
% frontend were crucial steps to maintain security and consistency across the MS.
% We also paid careful attention to the handling of guest users, giving them the option to merge
% their data with fully authenticated accounts without losing the assets they created.
%
% Overall, the work presented here lays the foundation for a robust multi-tenant experience in the PROCEED MS. It strikes a balance between introducing an entirely new concept—environments—and preserving the existing structure of the system so that both end users and developers can comfortably adapt to the new features.
%


% NOTE: maybe mention this??
% Nowadays software is so advanced in multi tenant platforms, companies have come to expect
% a lot of features from such software.
% Other than the role system this thesis implements the bare minimum to provide such an
% experience.

Building an application structure that supports multi-tenancy presents many challenges,
as demonstrated throughout this thesis.
In addressing these challenges within the PROCEED Management System (MS),
we introduced the concept of environments, isolated workspaces that
support both individual and collaborative use cases.

Significant architectural adjustments were required, particularly in restructuring how assets are managed.
Assets were adapted to explicitly associate them with specific environments, ensuring clear ownership and secure isolation.
To enhance organization and reflect user hierarchies more accurately, a flexible folder structure was implemented.
The preexisting role system was adapted to respect environment boundaries and folder
structures, ensuring isolation and further enhancing the ability to represent an organization's hierarchy.

Key challenges included restructuring the MS without disrupting existing functionality,
ensuring robust privacy and isolation between environments,
and adapting the role system to account for both folder and environment-specific permissions.
Moreover, the introduction of guest users added complexity to the authentication flows.

% We developed solutions that are closely aligned with the MS’s existing architecture.
% Leveraging CASL for backend permission enforcement and crafting simple, effective abstractions facilitated seamless frontend integration.

% While most outlined objectives were successfully achieved,
% there remain opportunities for future work.
% such as extending the folder structure to support additional asset types. 
% Nevertheless, this thesis establishes a robust foundation for multi-tenancy in the PROCEED
% MS.







\section{Outlook}
\label{cha:outlook}

% There are still many features that could be implemented to improve the expierience.
%
% - sharing documents to members that arent part of the organization
% - Support for user directives -> everyone doesn't have to create an account manually
%   - here it could be also possible that an organization is created and all accounts from
%     the user directive are automatically added 
% -

While most outlined objectives were successfully achieved,
there remain opportunities for future work,
such as extending the folder structure to support additional asset types.

Nevertheless, this thesis establishes a robust foundation for multi-tenancy in the PROCEED
MS allowing numerous additional features to be explored to further enhance user experience and productivity.
For example enabling real-time collaboration on assets such as processes.

Additionally, introducing user directives would significantly simplify user and organization management by enabling automated account creation and streamlined onboarding processes. 

