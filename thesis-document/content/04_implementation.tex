\chapter{Implementation}
\label{cha:implementation}

\section{Users}

We authenticate users by storing a JWT
\footnote{\url{https://www.rfc-editor.org/rfc/rfc7519.html}} cookie in their browser.
This cookie is then parsed by the MS backend and if the cookie is valid and it stores the
% TODO: is id the correct word here
id of an existing user, the user is considered authenticated.
If the user couldn't be authenticated, he is redirected to the sign in page.

\subsection{Guest Users}

Users that aren't signed in can choose to try the MS out as a guest, this doesn't require
the user to input any personal information.

For storing guest user data, one could take one of two approaches:
storing the data in the user's browser or storing it in the MS's database, alongside the
data of authenticated users.
Storing the data locally has two great benefits: 
The MS doesn't have to store data of users who might never return and
the MS would become less susceptible to an attack where the attacker tries to use up as
much space as possible in the MS's database.
However, this approach has one key downside, the MS would have to implement two storage
solutions and accordingly switch between them.
The added complexity of storing guest user's data locally isn't worth the benefits, so we
decided to store a guest user's data in the MS's database.

If a user chooses to try the MS out as a guest, a new user entry is created in the 

\subsection{Authenticated Users}


\begin{lstlisting}[
  language=json,
  style=codestyle,
  caption={CASL example},
]
{
    isGuest: false;
    emailVerifiedOn: Date | null;
    firstName?: string | undefined;
    lastName?: string | undefined;
    username?: string | undefined;
    image?: string | null | undefined;
    favourites?: string[] | undefined;
    id?: string | undefined;
    email?: string | undefined;
}
\end{lstlisting}

