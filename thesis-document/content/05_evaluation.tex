\chapter{Evaluation}
\label{cha:evaluation}

In this chapter we will evaluate the implementation of environments in the PROCEED MS,
for this we will go through the requirements defined in \ref{cha:tasklist}.
The primary goals of this thesis were to introduce environments as isolated workspaces for personal and organizational use,
enable efficient asset management through a hierarchical folder structure,
and adapt the MS's a role-based access control system to fit environments.
The following \ref{fig:quick-evaluation} table shows a quick rundown of the evaluation.


\begin{figure}[H]
	\centering

	\begin{tabular}{ | m{20em} | m{17em}| }
		\hline
		 Task & Status \\
     \hline
      1. Implement environments &  Partially implemented: folders only work for processes. \\
     \hline
      2. Personal and organization Environments &  Implemented \\
     \hline
      3. Adapt the MS' user management system to fit environments &  Implemented \\
     \hline
      4. Adapt the MS' role system to fit environments &  Implemented \\
     \hline
  %    1.a & x \\
		%  1.b &  \\
		%  1.c &  \\
		%  1.d &  \\
		%
		%  2.a &  \\
		%  2.b &  \\
		%  2.c &  \\
		%  2.d &  \\
		%  2.e &  \\
		%  2.f &  \\
		%  2.g &  \\
		%  3.a & \\
		%
		%  3.b.i &  \\
		%
		%  3.b.ii &  \\
		%  3.b.iii &  \\
		%  3.b.iv &  \\
		%  3.c &  \\
		%  3.d &  \\
		%
		%  4.a &  \\
		%  4.b &  \\
		%  4.c &  \\
		%  4.d &  \\
		%  4.e &  \\
		% ------------------------------------------------------------------------------
	\end{tabular}

	\caption{Evaluation of Task List.}
	\label{fig:quick-evaluation}
\end{figure}

According to \textbf{task 1} Environments are stored as entries in the MS's storage
solution with a unique id.
Every asset in the MS explicitly stores the id of the environment it belongs to.
Memberships to environments are stored in a separate table, where each entry has a user id
and an environment id,
later on we will explain how the access to assets is managed.
Furthermore, each environment has a root folder, which contains more folders and
processes.
Apart from root folders, both process and folders store the id of the folder they belong
to.

According to \textbf{task 2} environments store a flag to determine whether they are a
personal or an organization environment.
Personal environments are created when a user is created and organization environments can
be created by signed-in users.
Personal environments have a restricted feature-set, this is enforced by the permissions
system (task 4).

According to \textbf{task 3}, the user management was redesigned to accommodate
multi-tenancy,
users are now stored independently of the environments they belong to.
This allows for them to be part of multiple environments.
Users also have the option to sign in as a guest, which allows them to try the MS inside
a personal environment, without having to sign up, however they can not create of be part
of organization environments.

According to \textbf{task 4}, the MS's role system was adapted to fit environments and
folders. In its essence, roles work as they do before with two key differences:
Each role belongs to an environment and is only applied for assets inside that
environment, and roles can be scoped to folders, meaning that its permissions apply on all
descendants of the folder.
The role system was also leveraged to restrict the feature set of personal environments:
while personal environments don't have roles, the MS still uses the role system to manage
access to assets inside personal environments, it gives the owner of the environment all
permissions for the allowed features and restricts the rest.

\begin{enumerate}
	\item The MS has to support environments, isolated spaces where users can work on their
	      assets. \label{cha:tasklist:item-environments}
	      \begin{enumerate}
		      \item Every asset in the MS \ref{cha:relatedwork:proceed-assets} must be stored in only one environment.

		      \item Assets stored in one environment can only be accessed by members of that
		            environment.

		      \item Environments must have a hierarchical folder system to store assets.
		            \begin{enumerate}
			            \item Find a suitable abstraction to represent folders in a database
			                  that facilitates consistency after updates and is fast to query.

			            \item Ensure privacy between environments.

			                  % escribir aca ya que tiene que haber herencia
			                  % reescribir should be modified no es tan fuerte/
			                  % likely redundant

			                  % depending on the folder the asset is in.
		            \end{enumerate}

		            % NOTE: replace concurrently with eaesier word
		      \item The MS must be able to hold multiple environments and let users access them concurrently.

	      \end{enumerate}
	      % ---------------------------------------------------------------------------------------
	      % ---------------------------------------------------------------------------------------

	      %write about how there are personal and organization environments
	\item Implement personal and organization environments. While both are environments
	      and share common functionality as described in
	      \ref{cha:tasklist:item-environments}, they must behave differently in some
	      situations.
	      \begin{enumerate}
		      \item Personal environments can only have one member.

		      \item Personal environments can only store Processes and Folders.

		      \item Organization environments support all assets described in \ref{cha:relatedwork:proceed-assets}.

		      \item Organization environments can have a name and description.

		      \item Organization environments must support multiple members, i.e. multiple users can
		            work on the assets stored inside of it.

		      \item Organization environments must have a role system, where roles can be assigned to
		            users, to manage their access to assets.

		      \item Users of organization environments that have right permissions must be able to invite users
		            to the organization environment.
	      \end{enumerate}


	\item The MS's user management has to be adapted to fit environments: before the
	      implementation of this thesis, users where strictly tied to one instance of the MS,
	      meaning each time a new instance of the MS was created, a new user storage had to be
	      configured.
	      \begin{enumerate}
		      \item Users have to be global, meaning that they don't belong to any environment.

		      \item Implement Guest users.
		            \begin{enumerate}
			            \item Allow users to try the MS without signing in with their personal
			                  information.
			            \item Guest users can sign in with their personal information and become a normal
			                  user.
			            \item Guest users can transfer their assets to a normal user.

			            \item Guest users can not create or be part of organization environments.
		            \end{enumerate}

		            % TODO: does this belong here?
		      \item Every user has to have a personal environment.

		            % TODO: does this belong here?
		      \item Personal environments must be tightly coupled with users, i.e. when a user is
		            deleted so is his personal environment.
	      \end{enumerate}


	\item The MS's preexisting role system must be adapted to fit organization environments
	      and their folder structure:
	      The MS already has a role system in place to manage users' access to resources,
	      this has to be adapted to work with organization environments.
	      \begin{enumerate}
		      \item Roles must belong to only one organization environment.
		      \item The role system must be replicated for each environment, i.e. it works the
		            same as before, with the difference that it is now specific to an environment.
		            E.g. if a role allowed a user to manage all processes before the implementation of
		            environments, now, with the same role, he will be able to modify all processes
		            inside the organization environment the role belongs to.
		      \item Find a suitable permission inheritance model for roles based on the folder
		            structure of an environment (e.g. a user with a role in a parent
		            folder, has the same permissions in all subfolders).
		      \item Ensure roles are always enforced in the backend.
		      \item The frontend UI must adapt to a user's roles, by only showing options that the user has permission to do.
	      \end{enumerate}
\end{enumerate}

The following are non-functional requirements that have to be met for the implementation
of environments in the MS.
The goal of these is to ensure that the implementation is user-friendly and most
importantly developer-friendly, as many developers will have to work with the codebase in
the future.
This means that where possible, simple solutions should be favored over complex ones.

\begin{enumerate}
	\item Keep changes to the MS to a minimum.

	      % \item The implementation shouldn't be repetitive, the same functional components should be used for both personal and organization environments.

	\item The user interface for navigating and managing folders and environments should
	      be intuitive and easy to use.

	\item Prioritize developer experience by creating clear abstractions and APIs.
	      \begin{enumerate}
		      \item The same data structure and functions should be used for both personal and organization
		            environments where possible. E.g. the same function that creates a folder in a
		            personal environment should be used to create a folder in an organization environment.

		      \item Choose a simple data structure for the folder system, with straightforward
		            functions for modification.

		      \item Create simple abstractions for the backend code of the MS, that allow to
		            acknowledge a user's environment with minimal effort.

		      \item Create a simple abstraction for the frontend, that facilitates adapting
		            the Interface for each.
	      \end{enumerate}

	      % \item Environments should be easy to create, manage and delete.
	      %   \begin{enumerate}
	      %     \item The frontend should provide intuitive interfaces for creating and deleting organization environments.
	      %       // not functional
	      %     \item Provide well documented APIs to create, manage and delete environments both for the frontend and the backend.
	      %       needs more description. ldap
	      %     \item Environments should provide the option to import an existing user database.
	      %   \end{enumerate}
\end{enumerate}

