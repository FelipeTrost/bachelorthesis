\chapter{Evaluation}
\label{cha:evaluation}

In this chapter we will evaluate the implementation of environments in the PROCEED MS,
for this we will go through the requirements defined in \ref{cha:tasklist}.
The primary goals of this thesis were to introduce environments as isolated workspaces for personal and organizational use,
enable efficient asset management through a hierarchical folder structure,
and adapt the MS's a role-based access control system to fit environments.
The following \ref{fig:quick-evaluation} table shows a quick rundown of the evaluation.


\begin{figure}[H]
	\centering

	\begin{tabular}{ | m{20em} | m{17em}| }
		\hline
		 Task & Status \\
     \hline
      1. Implement environments &  Partially implemented: folders were only implemented for processes. \\
     \hline
      2. Personal and organization Environments &  Implemented \\
     \hline
      3. Adapt the MS' user management system to fit environments &  Implemented \\
     \hline
      4. Adapt the MS' role system to fit environments &  Implemented \\
     \hline
  %    1.a & x \\
		%  1.b &  \\
		%  1.c &  \\
		%  1.d &  \\
		%
		%  2.a &  \\
		%  2.b &  \\
		%  2.c &  \\
		%  2.d &  \\
		%  2.e &  \\
		%  2.f &  \\
		%  2.g &  \\
		%  3.a & \\
		%
		%  3.b.i &  \\
		%
		%  3.b.ii &  \\
		%  3.b.iii &  \\
		%  3.b.iv &  \\
		%  3.c &  \\
		%  3.d &  \\
		%
		%  4.a &  \\
		%  4.b &  \\
		%  4.c &  \\
		%  4.d &  \\
		%  4.e &  \\
		% ------------------------------------------------------------------------------
	\end{tabular}

	\caption{Evaluation of Task List.}
	\label{fig:quick-evaluation}
\end{figure}

According to \textbf{task 1} Environments are stored as entries in the MS's storage
solution with a unique id.
Every asset in the MS explicitly stores the id of the environment it belongs to.
Memberships to environments are stored in a separate table, where each entry has a user id
and an environment id,
later on we will explain how the access to assets is managed.
Furthermore, each environment has a root folder, which contains more folders and
processes.
Apart from root folders, both process and folders store the id of the folder they belong
to.

According to \textbf{task 2} environments store a flag to determine whether they are a
personal or an organization environment.
Personal environments are created when a user is created and organization environments can
be created by signed-in users.
Personal environments have a restricted feature-set, this is enforced by the role
system, this will be explained in more detail when we address task 4.
Users in organization environments can potentially use all the MS's features, this is
determined by the roles they have inside that environment.
The user that creates an organization environment is assigned the role of admin inside
that environment.
Access to assets in organization environments is managed by the role system described in
task 4,
the admin role has all permissions for all assets in the environment.
Organization environments can have multiple members, new members can be invited by other
members with the right permissions.

According to \textbf{task 3}, the user management was redesigned to accommodate
multi-tenancy,
users are now stored as records in the MS's storage solution independently of the environments they belong to.
This allows for them to be part of multiple environments.
Users also have the option to sign in as a guest, which allows them to try the MS inside
a personal environment without having to sign up.
Guest users are stored as other users, but with a flag that indicates that they are a
guest.

According to \textbf{task 4}, the MS's role system was adapted to fit environments and
folders. In its essence, roles work as they do before with two key differences:
Each role belongs to an environment and is only applied for assets inside that
environment, and roles can be scoped to folders, meaning that its permissions apply on all
descendants of the folder.
The role system was also leveraged to restrict the feature set of personal environments:
while personal environments don't have roles, the MS still uses the role system to manage
access to assets inside personal environments, it gives the owner of the environment all
permissions for the allowed features and restricts the rest.

\ref{cha:tasklist} also describes a set of non-functional requirements, these are harder
to evaluate than functional requirements. For these reason we surveyed other developers
that are working on the MS to get their opinion on the implementation of environments. 
The following questions were asked:


