\chapter{Foundations}
\label{cha:relatedwork}

To understand the further chapters,
it is essential to first establish a foundational understanding of 
external technologies, and the core architectural elements of the MS.
% the key technologies
% and architectural elements in the PROCEED Management System (MS).
This chapter briefly introduces the OAuth 2.0 standard,
then describes the MS' architecture and its role-based access control (RBAC) system.

\section{JSON Web Tokens (JWT)}

JWT is an open, industry-standard \cite{rfc7519} that defines a compact and self-contained way for securely transmitting information between parties as a JSON object.
This information can be verified and trusted because it is digitally signed.

A JWT consists of three parts separated by dots.
Each part is a Base64Url encoded JSON string.
The three parts are:

\begin{enumerate}
  \item \textbf{Header}: \
    The header typically consists of two parts: the type of the token, which is JWT, and the signing algorithm being used, such as HMAC SHA256 or RSA.

  \item \textbf{Payload}:
    The payload sections contains a JSON object, the contents of it are arbitrary. 

\item \textbf{Signature}:
  The signature is used to verify that the payload of the JWT hasn't been changed.
  To create the signature, the header and the payload are concatenated with a period,
    this string is then either hashed concatenated with a secret, or encrypted using a private key.
\end{enumerate}


Once an application receives a JWT, it can compute the signature with the header and payload,
if this matches the signature included in the JWT we can be sure that we issued the
JWT and that the payload hasn't been tampered with.

\section{OAuth 2.0 and OpenID Connect}
\label{cha:relatedwork:oauth}

OAuth 2.0 is an open standard for access delegation,
commonly used as a way for users to grant client applications access to their information on other applications.
OAuth 2.0 was born as a necessary security measure, to avoid sharing plaintext credentials between applications.
As outlined in \cite{rfcOAuth2} plaintext credential sharing presents many security risks:

\begin{enumerate}
	\item Applications are forced to implement password authentication, to support the sharing of plaintext credentials.
	\item Third party applications gain overly broad access to the user's account.
	\item Users cannot revoke access to specific third party applications.
	\item If any of the third party applications are compromised, the user's account is at risk.
\end{enumerate}

OAuth 2.0 addresses these issues by decoupling the client application from the role of the resource owner,
meaning that the client application will not get a full set of permissions to the user's account.
Instead of users handing their credentials to third party applications,
they can grant these applications limited access to their resources with an access token. 
This method avoids the need for the user to share their credentials with third-party applications.

\subsection{OAuth 2.0 Roles}

OAuth 2.0 defines four roles for participants in the protocol flow:

%TODO: rewrite this, it is looking too much like the original spec
\begin{enumerate}
  \item \textbf{Resource owner}: The entity that can grant access to a protected resource, typically this would be an end user of a web application.
  \item \textbf{Resource server}: The server hosting the protected resources.
  \item \textbf{Client}: The application requesting access to the protected resources.
	      OAuth 2.0 distinguishes between two types of clients: confidential and public clients.
	      Confidential clients are capable of keeping their credentials confidential, while public clients, like browser-based applications, cannot.
  \item \textbf{Authorization server}: The server that issues access tokens to the client after the resource owner has been successfully authenticated.
\end{enumerate}

The resource server and the authorization server can be the same entity, but they are not required to be.

% \subsection{Authorization Grants}
%
% %TODO: public and private client
% %TODO: authorization tokens
%
% Authorization Grants are credentials that are issued to clients, which can be exchanged for an access token.
% This access token can be used to access the protected resources on the resource server.
% OAuth 2.0 defines four authorization grants with different flows.

%TODO: footnote for http redirects or smthing

%\subsubsection{Implicit}
%\label{cha:relatedwork:oauth:implicit}
%
%% The implicit grant is a simplified version of the Authorization Code grant.
%The implicit grant is very helpful for public clients, as it doesn't require confidential client credentials.
%This is very helpful for browser-based clients, as they can't store confidential credentials securely.
%In the implicit grant users are redirected to the authorization server, where they
%authenticate themselves and authorize the client.
%Afterwhich the authorization server issues an access token directly to the client,
%this is done so with a HTTP redirect, where the access token is embedded in the redirect URL,
%this way the client can extract the access token from the URL.
%% The access token gets embedded in the redirect URL, after the resource owner was authenticated, which is then sent to the client.
%
%In this flow the resource owner only authenticates with the authorization server,
%thus never having to share his credentials with the client.
%
%Implicit grants have many security risks, as the access token is exposed in the URL and can be intercepted by a malicious attacker.
%This is why PKCE (Proof Key for Code Exchange) was later introduced as an addition to the implicit grant \cite{rfcPkce}.
%
% \subsubsection{Resource Owner Password Credentials}
% \label{cha:relatedwork:oauth:passwordcredentials}
%
% This grant type requires the resource owner to share his password credentials with the client.
% The resource owner's password credentials represent an authorization grant,
% which the client can exchange them for an access token.
% Even though this grant type requires the resource owner to share his credentials with the client,
% these are only used for one request and don't have to be stored.
%
%
% \subsubsection{Client Credentials}
% \label{cha:relatedwork:oauth:cleintcredentials}
%
% The client credentials grant is used when the client is the resource owner.
% Clients are typycally issued credentials, which they can use to authenticate themselves.
% Clients send these credentials to the authorization server and are issued an access token.

\subsubsection{Authorization Code}
\label{cha:relatedwork:oauth:authcode}

Of the possible OAuth 2.0 authorization mechanisms,
the Authorization Code grant is the most widely used for confidential clients (i.e., those capable of safely storing secrets).
The Authorization steps are as follows:

\begin{enumerate}

  \item \textbf{Client Redirects to Authorization Server}:
    The client first redirects the user to the Authorization Server’s authorization endpoint.
    This request includes information such as the client’s identifier, the requested scope of access,
    and the URL to which the Authorization Server should redirect once the user grants or denies permission.

  \item \textbf{User Authenticates and Grants Access}:
    The user logs in (if they are not already) in the Authorization Server and decides whether to grant the client’s request.
    If the user grants access, the Authorization Server returns only an authorization code to the client via the specified redirect URL.

  \item \textbf{Client Exchanges Code for Access Token}:
    After receiving the authorization code, the client sends it to the Authorization Server’s token endpoint,
    authenticating itself using its client credentials.
    In response, the Authorization Server returns an access token.
    The client then can use the access token to call APIs on the Resource Server.
\end{enumerate}

% Because the client must present its own client credentials (such as a client secret) to retrieve the access token, the Authorization Code grant reduces the risk of token exposure in situations like browser redirects.

% The Authorization Code grant is the most common grant type used in OAuth 2.0,
% it is similar to the implicit grant \ref{cha:relatedwork:OAuth:implicit}, as it also uses HTTP redirects and it doesn't require the resource owner to share his credentials with the client.
% In the authorization code grant, the client redirects the resource owner to the authorization server.
% There the resource owner authenticates himself and authorizes the client.
% afterwhich the authorization server redirects the resource owner back to the client with an authorization code.
% The client then authenticates itself with his confidential credentials on the authorization server and exchanges the authorization code for an access token.
% As the client needs confidential credentials, this flow is only suitable for confidential clients.
% The exact steps are shown in figure \ref{fig:OAuth:authcodeflow}.


\subsection{OpenID Connect}
\label{cha:relatedwork:oauth:openid}

% source: https://openid.net/specs/openid-connect-core-1_0.html

OpenID (OIDC for short) Connect is an identity layer built on top of the OAuth 2.0
\cite{OpenIDConnectCore}.
% While OAuth 2.0 focuses on authorization (granting clients access to resources),
% OIDC extends this to authentication.
% Since OIDC deals with authentication, I will call the resource owner the user from now on.
%
% OIDC leverages the Authorization Code flow (described in Section \ref{cha:relatedwork:OAuth:authcode}) of OAuth 2.0. However,
OIDC uses as its base the Authorization Code flow, 
and it introduces a new type of token called an ID Token.
This ID Token is a JSON Web Token (JWT) that contains minimal information about the authenticated
user.
Most importantly, the ID Token carries a Subject ID, which is a unique identifier
for the user in the Identity Provider's system.
The ID Token is sent alongside the Access token to the client.
Through the Subject ID the client can authenticate the user, since it came from a trusted
source, and sign them in to their platform.



% 3.2.1.  Implicit Flow Steps
% The Implicit Flow follows the following steps:
%
% Client prepares an Authentication Request containing the desired request parameters.
% Client sends the request to the Authorization Server.
% Authorization Server Authenticates the End-User.
% Authorization Server obtains End-User Consent/Authorization.
% Authorization Server sends the End-User back to the Client with an ID Token and, if requested, an Access Token.
% Client validates the ID token and retrieves the End-User's Subject Identifier.
% When using the implicit flow, the Id Token and, if requested, the Access Token
% are sent to the client after the user authenticated himself.
% After this the client could also use the Access Token to access 

% In the OIDC flow, after the client obtains the Authorization Code,
% it exchanges it for both an Access Token (as in OAuth 2.0) and an ID Token.
% The client can then validate the ID Token to ensure it's genuine and extract the user information contained within.


% 3.1.1.  Authorization Code Flow Steps
% The Authorization Code Flow goes through the following steps.
%
% Client prepares an Authentication Request containing the desired request parameters.
% Client sends the request to the Authorization Server.
% Authorization Server Authenticates the End-User.
% Authorization Server obtains End-User Consent/Authorization.
% Authorization Server sends the End-User back to the Client with an Authorization Code.
% Client requests a response using the Authorization Code at the Token Endpoint.
% Client receives a response that contains an ID Token and Access Token in the response body.
% Client validates the ID token and retrieves the End-User's Subject Identifier.


\section {MS Architecture}
\label{cha:ms-architecture}

% Before diving into the implementation details of environments, it is important to
% understand the architecture of the MS.
The MS is built using Next.js \footnote{\url{https://nextjs.org/}}, a React \footnote{\url{https://reactjs.org/}}
framework that allows for server-side rendering.
Although Next.js' architecture is different from traditional server-side rendered
applications and single-page applications,
for the purposes of this thesis,
it can be thought of as being split into a single-page frontend and a backend.
The frontend executes JavaScript code in the user's browser, and is
responsible for rendering the UI, handling user input and making requests to the backend.
The backend runs on a server and is responsible for handling requests from the frontend,
e.g. saving or querying data.

% TODO: too technical and probably not needed
% \subsection{Endpoints}
% \label{cha:ms-architecture:endpoints}
%
% Next.js implements RPC (Remote Procedure Call), by allowing us to write functions that run
% in the server, which can be imported in the frontend and called with serializable arguments.
% All files that contain these functions need to use the "use server"
% directive\footnote{\url{https://react.dev/reference/rsc/use-server}}, either inside the
% function, or at the top of the file.
% These functions will be called endpoints from now on.
%
% \begin{lstlisting}[
%   language=JavaScript,
%   style=codestyle,
%   caption={Example of a Zod schema and the corresponding TypeScript type.},
% ]
%   // server.js
%   function deleteProcess(environmentId, processId) {
%     "use server"
%
%     // delete process
%   }
%
%   // client.js
%   const button = document.getElementById('delete-process-button');
%   button.addEventLienvironmentId, processIdstener('click', (e) => {
%     const environmentId = e.target.dataset.environmentId
%     const processId e.target.dataset.processId
%     deleteProcess(environmentId, processId);
%   });
% \end{lstlisting}

\subsection{PROCEED MS' Storage Solution}
\label{cha:relatedwork:proceed-storage}

The PROCEED MS doesn't use a database management system (DBMS) like MySQL or PostgreSQL to store its data.
Instead, it stores its data in multiple
JSON \footnote{https://datatracker.ietf.org/doc/html/rfc8259} files.
While these files allow for a very flexible data structure, the MS only stores one array of
objects per file,
each with a fixed schema.
This makes each file comparable to a table in a relational database.
Each object will be called an entry from now on.
The MS uses Zod \footnote{\url{https://zod.dev/}} to enforce that the data that is being
stored follows a specific structure. 
Zod is a schema declaration and validation library, it allows the MS to define the shape
of JSON-serializable data.
For purposes of simplicity, when we talk about a schema, instead of showing the code that
describes the schema, we will show the typescript type that is inferred from the schema.

 \begin{lstlisting}[
   language=JavaScript,
   style=codestyle,
   caption={Example of a Zod schema and the corresponding TypeScript type.},
 ]
 import { z } from 'zod';

 const UserSchema = z.object({
   id: z.string(),
   username: z.string(),
   image: z.string().optional(),
 })

 // TypeScript type that satisfies the UserSchema
 type User = {
     id: string;
     username: string;
     image?: string | undefined;
 }
 \end{lstlisting}

 % \subsection{Data storage}
 % \label{cha:ms-architecture:data-storage}

% % TODO: footnote or something for json
%
% The MS doesn't use a database management system, instead it stores all data in JSON files.
% Each file can be seen as a table in a traditional relational database.
% Even though this approach allows for unstructured data, the MS uses

% TODO !!
\subsection{PROCEED's Assets}
\label{cha:relatedwork:proceed-assets}

Assets are objects that users can create and manage through the MS' interface.
They are the core "product" that PROCEED offers, i.e. the focal point of its value
proposition,
in contrast to management assets, which regulate how a user can use the MS.
When referring to assets, we are only talking about the core features of the MS,
not about objects that aid in the usage of the MS, like Roles .e.g.,
which only help with managing access to assets.
Currently, the MS supports the following assets:

\begin{enumerate}
  \item \textbf{Processes, Project and Templates}: These assets store BPNN at their core.
	%       % TODO: description for task
	% \item Task:
  \item \textbf{Machine}: Asset that represents a server running Distributed Process Engine. 
    % The machine is used to manage properties of the server.
	      % NOTE  maybe remove execution
  \item \textbf{Execution}: An execution represents a process that is being executed distributedly.
\end{enumerate}

Furthermore, the MS implements management assets, which are used to control how users interact with the MS.
These are considered assets because users can directly manage them.
Currently, the MS includes the following management assets:

\begin{itemize}
  \item \textbf{Role}: roles are used to manage how users can access assets.
  \item \textbf{RoleMapping}: role mappings are used to assign roles to users.
  \item \textbf{User}: represents a user's personal information, e.g. name, username and email.
\end{itemize}


\section{PROCEED's Role System}
\label{cha:relatedwork:proceedroles}

The PROCEED MS uses a Role-Based Access Control %(RBAC) 
system to manage user authorization and determine what actions a user can perform.
Roles can be seen as bundles of permissions, which are granted to users.
A user can have multiple roles and all the permissions of the roles are additively
combined. That is, by adding a permission, a user can never do less than before.
Typically, roles are assigned to users based on their job function.
%% Advantages 
%% - One role serves multiple people
%% - roles don't change ofte
%% - roles are easier to manage than individual permissions
RBAC can be advantageous since roles can be assigned to multiple users and
don't change often, making them easier to manage than individual permissions.

\subsection{MS' Role System Terminology}
\label{cha:relatedwork:proceedroles:terminology}

The following terms are important to understand the role system in the MS:

\begin{itemize}
  \item \textbf{Resource}: A resource is any protected entity in the management system, that can be
    accessed by users. Resources can be either assets or management assets.
    % assets \ref{cha:relatedwork:proceed-assets}, but they don't have to be.
  \item \textbf{Action}: An action is a specific operation that can be performed on a resource, e.g. view, update, create, delete.
  \item \textbf{Permission}: A permission is a tuple consisting of a resource type and a list of
    actions, which specifies that a user can perform the actions on the resource instances.
    Optionally a permission can have conditions that have to be met the by resource instances,
    for the user to be able to perform the actions.
  \item \textbf{Role}: A role is a set of permissions. Roles can be assigned to users, which then
    inherit the role's permissions. Roles can have expiration dates, after which all
    permissions are revoked.
\end{itemize}

\subsection{MS' Resources and Actions}
\label{cha:relatedwork:proceedroles:ms-resources-actions}

The following are the resource types that are used in the PROCEED MS:
\lstinline{Process},
\lstinline{Project},
\lstinline{Template},
% \lstinline{Task},
\lstinline{Machine},
\lstinline{Execution},
\lstinline{Role},
\lstinline{User},
% \lstinline{Setting},
\lstinline{RoleMapping}.

These are the actions that can be performed on these resources:
\lstinline{none},
\lstinline{view},
\lstinline{update},
\lstinline{create},
\lstinline[keywords={}]{delete}.

\subsection{Role Mappings}
\label{cha:relatedwork:proceedroles:role-mappings}

\lstinline{RoleMappings} are a management asset, that is used to assign roles to users.
\lstinline{RoleMappings} store a user identifier ID and a role ID.

\subsection{MS' Roles in CASL}
\label{cha:relatedwork:proceedroles:casl}

The PROCEED MS uses CASL \footnote{https://casl.js.org/v6/en/} to implement Roles. 
CASL is an isomorphic authorization JavaScript library.
To enforce authorization CASL has \textit{abilities}, which are assigned to users.
Abilities expose functions to check wether a user can perform an action on a resource.
Abilities are made up of rules, which are defined by three parameters: user action, subject,
% fields,
conditions.
User actions and subjects are analogous to actions and resources \ref{cha:relatedwork:proceedroles:terminology}.

CASL differentiates between subject type and subject instance.
A subject instance is a specific instance of a subject type, e.g. a specific process
users are working on, is an instance of the resource type "Process".

% Fields are used to specify which fields of a resource instance an action can be performed
% on, e.g. a user can update a process's name, but not its id, or creation date.

Conditions are used to specify additional conditions that must be met by a resource
instance, for a user to be able to perform an action on it. E.g. a user can only update a
process if he created it.

\begin{lstlisting}[
  language=Javascript,
  style=codestyle,
  caption={CASL example},
]
import { defineAbility } from '@casl/ability';

class User {
  constructor(id) {
    this.id = id;
  }
}

class Process {
  constructor(user, name) {
    this.authorId = user.id;
    this.createdOn = new Date();
    this.name = name
  }
}

function abilityForUser(user){
  return defineAbility((can, cannot) => {
    can('delete', 'User', {id: user.id});

    can('update', 'Process', ['name'], {authorId: user.id});
  });
}

const user1 = new User(1);
const user1Ability = abilityForUser(user1);
const user1Process = new Process(user1, 'some process');

const user2 = new User(2);
const user2Ability = abilityForUser(user2);

user1Ability.can('update', 'Process'); // true
user1Ability.can('update', user1Process, 'name'); // true
user1Ability.can('update', user1Process, 'createdOn'); // false

user1Ability.can('delete', user1); // true
user1Ability.can('delete', user2); // false

user2Ability.can('update', 'Process'); // true
user2Ability.can('update', user1Process); //false
\end{lstlisting}

If there exist any possible resource instance, for which a user has permission to perform an action,
then the user has permission to perform the action on the resource type.
E.g if a user has permission to view some process in the MS, then he has permission to
view the resource type "Process".

