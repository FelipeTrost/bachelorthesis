\chapter*{Zusammenfassung}
\label{cha:zusammenfassung}

Diese Arbeit untersucht die Integration von Multi-Tenancy-Funktionalität in Cloud-Anwendungen am Beispiel des PROCEED Management Systems (MS).
Aktuell hat das PROCEED MS keine Unterstützung für Multi-Tenancy,
wodurch kollaborative Arbeitsabläufe nicht möglich sind.
Um diese Einschränkung zu beheben, wurden isolierte Umgebungen eingeführt,
namens \textit{Environments},
die es mehreren Nutzern oder Organisationen ermöglichen, ihre digitalen Güter innerhalb klar abgegrenzter,
sicherer Arbeitsbereiche gemeinsam zu verwalten.

Eine hierarchische Ordnerstruktur wurde entwickelt, um hierarchische Strukturen von
Organisationen abzubilden und das Management von digitale Güter zu verbessern.
Das bestehende rollenbasierte Zugriffskontrollsystem innerhalb vom PROCEED MS wurde ebenfalls erweitert, 
um \textit{Environments}- und ordnerspezifische Berechtigungen einzubeziehen.

Schwerpunkte der Umsetzung waren die Restrukturierung der digitale Güter- und Nutzermanagements sowie
die Etablierung klarer und effizienter Datenbankschematas.
Die entwickelte Lösung unterstützt sowohl persönliche also auch organisatorische Umgebungen und erlaubt es Nutzern,
leicht zwischen unabhängigen Projekten und kollaborativen Aufgaben zu wechseln.

Die Evaluation bestätigt die erfolgreiche Umsetzung der definierten funktionalen und nicht-funktionalen Anforderungen.
